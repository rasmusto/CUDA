%LATEX Header
\documentclass[11pt, twocolumn]{article}
\usepackage[letterpaper]{geometry}
\geometry{top=1.0in, bottom=1.0in, left=1.0in, right=1.0in}
\usepackage{times}
\usepackage{amssymb,amsmath}

% Title Page
\title{CUDA Implementation of Jacobi Relaxation}
\author{Torben Rasmussen, mawolfe, wehland, ssharma, neeraja}
\date{7/20/2011}

\begin{document}
\maketitle
\section{Abstract}
\section{Introduction} %background/context, the idea, summary of results
Over the past few years GPU’s which originally were used for high graphics applications, have now become GP-GPU’s (general purpose GPU’s).
Since almost 6 years now researchers have looked into the potential of GPU’s to do parallel mathematical applications (GPU computing), and hence enter CUDA.
CUDA is Nvidia’s parallel computing architecture, which gives the programmer an interface to the under lying GPU’s.
A GPU can offer a very high computational rate if the algorithm is well-suited for the device.
One of those applications is matrix computation.
Matrix applications like Jacobi relaxation are interesting to work on a parallel computing architecture.
Hence is a well suited application to work with on GP-GPU’s.
The rest of our report is organized in the following way.
The next section talks about our project idea, design and analysis of solution to our problem.
Section 3 shows the actual implementations followed by the results achieved.
In section 4 we discuss work related to ours and finally conclude in section 5.
We had limited goals due to shortage of time and hence we discuss future work in section 6.
    \subsection{Background} %CUDA, math

\section{Our Project}
    \subsection{Requirements} %hardware, software
    \subsection{Analysis} %how do we do testing, what are we measuring, etc.
    \subsection{Design} %what we did, how did we improve it (performance tuning).  input matrix size, kernel code (including block size), etc.

\section{Results} %prove the idea is good
Till now we have talked about the algorithm and design of our problem. In this section we show the
actual implementation and the results we got using these optimizations. We actually worked on two
levels of optimizations on [reference], one of those is the one kernel optimizations and the other one is
maximizing the throughput according to the array size.
    \subsection{implementation 1}
    Our first optimization is to implement Jacobi relaxation in CUDA with one kernel. In [reference]
    Michael has implemented Jacobi relaxation in CUDA using two kernel calls per iteration. Instead as an
    optimization we use a single kernel call per iteration reducing the over head to initiate an extra kernel
    each time. The reason for using two kernels initially was to reduce the change values across the blocks
    to one single value. These change values were reduced from one change value per thread to a change
    value per block, in the first kernel 'jacobikernel'. In the second kernel with fewer numbers of threads
    and one single block those values are reduced to one single change value. In our implementation we
    used the existing threads to do the additional work of further reducing the per block change values to
    a single value. With this optimization we were able to reduce the calculation time to 10%. However we
    could not reduce it much further because though here we are reducing the over head of initialization of
    a new kernel and creating new threads, but in our implementation there are other threads in the end
    which sit idle and do not have much work to do. However this is a possible way of optimization which
    did give us positive results. In the table below we have shown a few array sizes, their performance with

    one kernel (our implementation), performance with two kernels (as in [reference]) and the performance
    improvement of one kernel, over two kernel implementation.

    \begin{tabular}{ l l l l l l l l }
    Array Size  & 1kernel total time   & 2kernel  total time  & 1kernel calculation time    & 2kernel  calculation time   & Performance improvement on total time   & Performance improvement on calculation time & Percentage improvement in performance on calculation time of 1kernel over 2 kernel \\
    130x130 & 0.108135    & 0.106293    & 0.009158    & 0.010240    & 0.98296574  & 1.11814807  & 10.5  \\
    146x146 & 0.104811    & 0.106242    & 0.010526    & 0.011699    & 1.01365315  & 1.11143834  & 10    \\
    162x162 & 0.131627    & 0.115655    & 0.014078    & 0.015089    & 0.87865711  & 1.07181418  & 6.7   \\
    178x178 & 0.115519    & 0.118052    & 0.018908    & 0.020070    & 1.02192713  & 1.06145547  & 5.7   \\
    194x194 & 0.119967    & 0.119893    & 0.021031    & 0.022331    & 0.99938316  & 1.06181351  & 5.8   \\
    210x210 & 0.125773    & 0.128649    & 0.024906    & 0.026340    & 1.02286659  & 1.05757649  & 5.4   \\
    226x226 & 0.128909    & 0.130154    & 0.029529    & 0.030782    & 1.00965798  & 1.04243286  & 4     \\
    242x242 & 0.137315    & 0.150276    & 0.034656    & 0.037037    & 1.09438881  & 1.06870383  & 6.4   \\
    \end{tabular}

    \subsection{implementation 2}
    \subsection{summary} % of results with graphs/tables/etc.

\section{Related Work} %prove the idea is new? It really isn't, but it's our take on existing software.  This section may be small.
    \subsection{michael wolfe!!}
    \subsection{mention} %other papers we read in class?

\section{Conclusion} %repeat idea, summarize key results

\section{Future Work} %list the things you wanted to do but couldn't finish in time for this paper
\section{Acknowledgements} %mention wolfe et al
\end{document}
